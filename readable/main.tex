\documentclass[a4paper]{article}
\usepackage[utf8]{inputenc}

\title{AMP - Accelerated Mobile Pages}
\author{Enrico Salvucci}
\date{18 Maggio 2018}

\usepackage{natbib}
\usepackage{graphicx}
\usepackage{hyperref}

\begin{document}

\maketitle

\newpage

\tableofcontents

\newpage
\section{Preface}

We know the matter about User Experience: Who ever complained about a train late? and who ever complained about a web page loading time and, eventually, left it due to a long wait?

Today we want all "hic et nunc", istantly.

Many big Web Agencies took action to solve this problem and reduce the load time of a web page. Different approaches could be found: low level or high level approach.

\subsection{Approaches}
\begin{itemize}
    \item Low level
    \begin{itemize}
        \item Google SPDY\\
        https://developers.google.com/speed/spdy/
        \item HTTP/2.0
    \end{itemize}

    \item High level
    \begin{itemize}
        \item Polaris (MIT)
        \item Facebook Instant Articles (Facebook)\\
        \item Apple News Format (Apple)
        \item AMP (Google)
    \end{itemize}
\end{itemize}

\newpage
\subsection{Google AMP}
It's a project started in 2016 by Google which promote a fast load of a web page visited on mobile.
AMP grows on three base components:
\subsubsection{AMP HTML}
    \begin{itemize}
        \item Some deprecated html tags in AMP
        \begin{itemize}
            \item \textless base \textgreater
            \item \textless frame \textgreater
            \item \textless frameset \textgreater
            \item \textless object \textgreater
            \item \textless param \textgreater
            \item \textless applet \textgreater
            \item \textless embed \textgreater
            \item \textless form \textgreater Support coming in the future.
            \item \textless input \textgreater
            \item \textless script \textgreater
        \end{itemize}
        \item Replaced html tags in AMP
        \begin{itemize}
            \item \textless img \textgreater -\textgreater amp-img
            \item \textless audio \textgreater -\textgreater amp-audio
            \item \textless video \textgreater -\textgreater amp-video
            \item \textless iframe \textgreater -\textgreater amp-iframe
        \end{itemize}
        \item Granted html tags in AMP
        \begin{itemize}
            \item \textless button \textgreater
            \item \textless style \textgreater
            \item \textless link \textgreater Support coming in the future.
            \item \textless meta \textgreater
            \item \textless a \textgreater
            \item \textless svg \textgreater
        \end{itemize}
    \end{itemize}

    \subsubsection{AMP JS (AMP core)}
    %
    % The Idea was not no Javascript, the idea was "Good user experience"
    %
    % Web workers --> They are a way to have separated shared nothing threads running your javascript
    % The main thing about the workers is they don't actually have access to the page DOM by default
    % Everything you do in the DOM cannot block the main thread
    AMP allows only asynchronous script. The Idea is not no Javascript, the idea is "Good user experience".
    Javascript code could have unpredictable behaviour and in such way you avoid late page rendering and do not block DOM construction.
    
    %Le dimensioni delle risorse esterne, come per esempio immagini, ads pubblicitari ecc.., devono essere specificate staticamente in px; in questo modo AMP può renderizzare la pagina con i suoi contenuti e solo dopo "popolare" lo spazio vuoto con le risorse (più lente a caricarsi rispetto al testo).\\
    In the next future Javascript will be allowed on AMP pages (annunced in February at AMP Conf 2018) through the use of Web Workers (which can't touch the DOM by default).
    
    \subsubsection{Google AMP Cache}
    Google AMP employ a proxy system to cache contents and preload pages before their will be queried.
    
    \subsection{Validation}
    Another important AMP feature is validatidation system, which verifies the compliance of an AMP page according to the language specifics.

    %
    % Problema delle pagine AMP che puntano a google.com
    % "risolto" tramite il web packaging (che vogliono rendere in futuro uno standard).
    % creo un pacchetto con tutti i contenuti necessari alla AMP page (immagini, html, e quant'altro serva per mostrare la pagina..), lo firmo con una chiave crittografica e lo invio a qualunque google cache.
    % Quando la cache fornisce la pagina e i contenuti il browser riconosce che l'identita' di quel pacchetto viene da un altro dominio. Cosi' viene mostrato il dominio originale e non il google.com

\section{References}
\begin{itemize}
    \item https://ampbyexample.com/
    \item https://www.ampproject.org/it/docs/media/amp\_replacements
    \item https://www.ampproject.org/it/learn/about-how/ (in basso)
    \item \href{http://www.wired.com/2016/02/googles-amp-speeding-web-changing-works/}{http://www.wired.com/2016/02/googles-amp-speeding-web-changing-works/}
    \item \href{https://www.ampproject.org/docs/fundamentals/how_cached}{How AMP pages are cached}
    \item https://techcrunch.com/2018/03/08/google-promises-publishers-an-alternative-to-amp/
    \item https://www.theregister.co.uk/2017/05/19/open\_source\_insider\_google\_amp\_bad\_bad\_bad/
    \item https://www.ampproject.org/docs/fundamentals/converting/resolving-errors
    \item https://www.ampproject.org/docs/fundamentals/how\_cached
    \item https://www.youtube.com/watch?v=ol-YxPw66Kc (AMP Conf 2018)
    \item https://www.theverge.com/2018/3/8/17095078/google-amp-accelerated-mobile-page-announcement-standard-web-packaging-urls
\end{itemize}

\section{License}

Document written in \LaTeX\ and released under \textbf{\href{http://creativecommons.org/licenses/by-sa/4.0/}{Creative Commons - Attributions, Share-alike 4.0}}\\

\includegraphics[height=0.8cm]{images/cc.png}

Sources on \textbf{\url{https://github.com/esalvucci/AMP-talk}}
\end{document}


\documentclass{article}
\usepackage[utf8]{inputenc}

\title{AMP - Accelerated Mobile Pages}
\author{Enrico Salvucci}
\date{March 2016}

\usepackage{natbib}
\usepackage{graphicx}

\begin{document}

\maketitle

\section{Introduzione}

Problema: Quante volte avete lasciato un sito web perché stava impiegando un eternità a caricare?

il 47 per cento degli utenti di Internet si aspettano che un sito web carichi in meno di due secondi.
Durante i periodi di punta del traffico, il 75\% dei consumatori sono disposti a visitare siti concorrenti invece di attendere che una pagina finisca di caricare, quando è lenta.

I giganti del web (e non solo) si sono mossi per affrontare questo aspetto e trovare un modo per ridurre i tempi di caricamento delle pagine web visitate da Mobile.
Google ha posto grande attenzione alla questione mobile anche penalizzando nei risultati di ricerca (da circa un anno) i siti non "responsive".

\subsection{Approcci}
Esistono in particolare due approcci per favorire il caricamento di pagine web da mobile.
\begin{itemize}
    \item Basso livello (indirettamente)
    \begin{itemize}
        \item Google SPDY\\ 
        SPDY (pronounced "SPeeDY") is a networking protocol whose goal is to speed up the web.\\
        SPDY augments HTTP with several speed-related features that can dramatically reduce page load time\\
        https://developers.google.com/speed/spdy/
        \item HTTP/2.0
    \end{itemize}

    \item Alto livello
    \begin{itemize}
        \item Polaris (MIT)
        \item Facebook Instant Articles (Facebook)\\
        Oltre a garantire un caricamento molto rapido delle pagine dei siti "supporters", costringe a rimanere all ’interno dell ’ecosistema Facebook.
FA offre anche possibilità di interazione (condivisione, commento, like ecc..) con il contenuto che si stà visualizzando.
%% http://www.tsw.it/digital-marketing/content-marketing/abbiamo-provato-facebook-instant-articles-cose-e-come-funziona/
Facebook ha anche annunciato un plugin per wordpress. Il plugin ha lo scopo di aiutare la pubblicazione dei contenuti all’interno di Instant Articles
%% http://www.webnews.it/2016/03/08/facebook-instant-articles-wordpress/-

        \item Apple News Format (Apple)
        \item AMP (Google)
    \end{itemize}
\end{itemize}

\subsection{Google AMP}
AMP è un "dialetto" implementato da Google per favorire e velocizzare il caricamento delle pagine web richieste tramite mobile.

\begin{itemize}
    \item AMP HTML
    \item AMP JS (il cuore di AMP)
    \item Google AMP Cache\\
    Sistema di consegna dei contenuti basato su proxy.\\
    Google AMP Cache recupera le pagine AMP HTML, le memorizza in cache e ne ottimizza automaticamente le performance
    
    E' importante anche il sistema di validazione, tramite il quale si verifica che la pagina AMP html rispetti le specifiche e funzioni effettivamente.
    

\end{itemize}


\bibliographystyle{plain}
\bibliography{references}
\end{document}

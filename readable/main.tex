\documentclass{article}
\usepackage[utf8]{inputenc}

\title{AMP - Accelerated Mobile Pages}
\author{Enrico Salvucci}
\date{March 2016}

\usepackage{natbib}
\usepackage{graphicx}

\begin{document}

\maketitle

\section{Introduzione}

Problema: Quante volte avete lasciato un sito web perché stava impiegando un eternità a caricare?

il 47 per cento degli utenti di Internet si aspettano che un sito web carichi in meno di due secondi.
Durante i periodi di punta del traffico, il 75\% dei consumatori sono disposti a visitare siti concorrenti invece di attendere che una pagina finisca di caricare, quando è lenta.

I giganti del web (e non solo) si sono mossi per affrontare questo aspetto e trovare un modo per ridurre i tempi di caricamento delle pagine web visitate da Mobile.
Google ha posto grande attenzione alla questione mobile anche penalizzando nei risultati di ricerca (da circa un anno) i siti non "responsive".

\subsection{Approcci}
Esistono in particolare due approcci per favorire il caricamento di pagine web da mobile.
\begin{itemize}
    \item Basso livello (indirettamente)
    \begin{itemize}
        \item Google SPDY\\ 
        SPDY (pronounced "SPeeDY") is a networking protocol whose goal is to speed up the web.\\
        SPDY augments HTTP with several speed-related features that can dramatically reduce page load time\\
        https://developers.google.com/speed/spdy/
        \item HTTP/2.0
    \end{itemize}

    \item Alto livello
    \begin{itemize}
        \item Polaris (MIT)
        \item Facebook Instant Articles (Facebook)\\
        Oltre a garantire un caricamento molto rapido delle pagine dei siti "supporters", costringe a rimanere all ’interno dell ’ecosistema Facebook.
FA offre anche possibilità di interazione (condivisione, commento, like ecc..) con il contenuto che si stà visualizzando.
%% http://www.tsw.it/digital-marketing/content-marketing/abbiamo-provato-facebook-instant-articles-cose-e-come-funziona/
Facebook ha anche annunciato un plugin per wordpress. Il plugin ha lo scopo di aiutare la pubblicazione dei contenuti all’interno di Instant Articles
%% http://www.webnews.it/2016/03/08/facebook-instant-articles-wordpress/-

        \item Apple News Format (Apple)
        \item AMP (Google)
    \end{itemize}
\end{itemize}

\subsection{Google AMP}
AMP è un "dialetto" implementato da Google per favorire e velocizzare il caricamento delle pagine web richieste tramite mobile.

\subsubsection{AMP HTML}
    \begin{itemize}
        \item I tag html "deprecati" da AMP
        \begin{itemize}
            \item \textless base \textgreater
            \item \textless frame \textgreater
            \item \textless frameset \textgreater
            \item \textless object \textgreater
            \item \textless param \textgreater
            \item \textless applet \textgreater
            \item \textless embed \textgreater
            \item \textless form \textgreater Support coming in the future.
            \item \textless input \textgreater
            \item \textless script \textgreater
        \end{itemize}
        \item I tag html "rimpiazzati" da AMP
        \begin{itemize}
            \item \textless img \textgreater - rimpiazzato da amp-img
            \item \textless audio \textgreater - rimpiazzato da amp-audio
            \item \textless video \textgreater - rimpiazzato da amp-video
            \item \textless iframe \textgreater - rimpiazzato da amp-iframe
        \end{itemize}
        \item I tag html permessi in  AMP
        \begin{itemize}
            \item \textless button \textgreater
            \item \textless style \textgreater
            \item \textless link \textgreater Support coming in the future.
            \item \textless meta \textgreater
            \item \textless a \textgreater
            \item \textless svg \textgreater
        \end{itemize}
    \end{itemize}

    \subsubsection{AMP JS (il cuore di AMP)}
    AMP permette l'esecuzione di soli script asincroni (ovvero che vengono richiamati solo una volta che tutto il resto della pagina è stato caricato). Questo permette di evitare ritardi e peggiorare i tempi di caricamento.\\
    \\
    Le dimensioni delle risorse esterne, come per esempio immagini, ads pubblicitari ecc.., devono essere specificate staticamente in px; in questo modo AMP può renderizzare la pagina con i suoi contenuti e solo dopo "popolare" lo spazio vuoto con le risorse (più lente a caricarsi rispetto al testo).\\
    
    \subsubsection{Google AMP Cache}
    Sistema di consegna dei contenuti basato su proxy.\\
    Google AMP Cache recupera le pagine AMP HTML, le memorizza in cache e ne ottimizza automaticamente le performance
    
    E' importante anche il sistema di validazione, tramite il quale si verifica che la pagina AMP html rispetti le specifiche e funzioni effettivamente.

\subsection{Wordpress plugin(s)}

\bibliographystyle{plain}
\bibliography{references}
\end{document}
